%%%%%%%%%%%%%%%%%
% This is an example CV created using altacv.cls (v1.1, 21 November 2016) written by
% LianTze Lim (liantze@gmail.com), based on the 
% Cv created by BusinessInsider at http://www.businessinsider.my/a-sample-resume-for-marissa-mayer-2016-7/?r=US&IR=T
% 
%% It may be distributed and/or modified under the
%% conditions of the LaTeX Project Public License, either version 1.3
%% of this license or (at your option) any later version.
%% The latest version of this license is in
%%    http://www.latex-project.org/lppl.txt
%% and version 1.3 or later is part of all distributions of LaTeX
%% version 2003/12/01 or later.
%%%%%%%%%%%%%%%%

%% If you want to use \orcid or the
%% academicons icons, add "academicons"
%% to the \documentclass options. 
%% Then compile with XeLaTeX or LuaLaTeX.
% \documentclass[10pt,a4paper,academicons]{altacv}
\documentclass[10pt,a4paper]{altacv}

%% AltaCV uses the fontawesome and academicon fonts
%% and packages. 
%% See texdoc.net/pkg/fontawecome and http://texdoc.net/pkg/academicons for full list of symbols.
%% When using the "academicons" option,
%% Compile with LuaLaTeX for best results. If you
%% want to use XeLaTeX, you may need to install
%% Academicons.ttf in your operating system's font %% folder.

\usepackage{hyperref}
% Change the page layout if you need to
\geometry{left=1cm,right=9cm,marginparwidth=6.8cm,marginparsep=1.2cm,top=1cm,bottom=1cm}

% Change the font if you want to.

% If using pdflatex:
\usepackage[utf8]{inputenc}
\usepackage[T1]{fontenc}
\usepackage[default]{lato}

% If using xelatex or lualatex:
% \setmainfont{Lato}

% Change the colours if you want to
\definecolor{VividPurple}{HTML}{2E64FE}
\definecolor{SlateGrey}{HTML}{2E2E2E}
\definecolor{LightGrey}{HTML}{666666}
\colorlet{heading}{VividPurple}
\colorlet{accent}{VividPurple}
\colorlet{emphasis}{SlateGrey}
\colorlet{body}{LightGrey}

% Change the bullets for itemize and rating marker
% for \cvskill if you want to
\renewcommand{\itemmarker}{{\small\textbullet}}
\renewcommand{\ratingmarker}{\faCircle}



\begin{document}
\photo{3.5cm}{12.jpg}
\name{Oumayma BELLILA }
 

\personalinfo{%
  % Not all of these are required!
  % You can add your own with \printinfo{symbol}{detail}
  \printinfo{3rd grade student at the higher school of statistics and data analysis (ESSAIT)}\\
\printinfo{looking for an end-of-studies project about artificial intelligence, starting Februry 2020}\\
\vspace{0.2cm}
  
   \email{oumaymabellila@gmail.com} \quad
  \phone{+216 27 422 031}\\  
  \vspace{0.2cm}
  \location{Ariana, Tunis}     {Age: 24}\\
  \vspace{0.2cm}
   \faLinkedinSquare \quad { \href{https://www.linkedin.com/in/oumayma-bellila-81468b178/}{Oumayma Bellila} }
  \quad \faGithubSquare \quad {\href{https://github.com/May-B}{May-B}}
%   \orcid{orcid.org/0000-0000-0000-0000} % Obviously making this up too. If you want to use this field (and also other academicons symbols), add "academicons" option to \documentclass{altacv}
}

%% Make the header extend all the way to the right, if you want. Extend the right margin by 8cm (=6.8cm marginparwidth + 1.2cm marginparsep)
\begin{adjustwidth}{}{-8cm}
\makecvheader
\end{adjustwidth}

%% Provide the file name containing the sidebar contents as an optional parameter to \cvsection.
%% You can always just use \marginpar{...} if you do
%% not need to align the top of the contents to any
%% \cvsection title in the "main" bar.
\cvsection[page1sidebar]{professional experience}

\cvevent{}{Internship | Pasteur Institute of Tunis}{June - July 2019}{Monplaisir, Tunis}
\begin{itemize}
\item manipulating/joining genomic databases (big data, text mining)
\item extracting a list of micro ARN in relation with type-2-diabetes (enviroment : python)
\item constructing a predictive model for breast cancer based on the genetic variants génétiques of the patients (machine learning on R)
\end{itemize}

\cvevent{}{Internship | One To One for Research and POLLING}{August 2018} {Mutuelle Ville, Tunis}

Programming Questionnaires on CSpro 




\cvevent{}{Teleoperator | Phone Academy Call Center}{October - November 2017}{Lafayette, Tunis}
\begin{itemize}
\item Collecting responses for a phone survey on the usage of renewable energies
\end{itemize}



\cvsection{acedemic / free-lance projects}

\cvevent{}{Image Classification to detect cancerous biopsies (Deep Learning)}{December 2018} {}
constructing a convlutional neural network (CNN) that recognizes images of cancerous biopsies on Python\\
\textbf{tools :} the Keras and Tensorflow libraries on Python


\cvevent{}{Machine Learning project for insurance covering}{May 2019} {}
constructing 3 models on R (logit model, decision tree and neural network) to predict if an indivdual will have full insurance coverage or not.


\cvevent{}{JAVA application with Swing}{October 2019}{}
Programming an app following the MVC model, connected to an external database (Oracle) and that manages user input.

\cvevent{}{Cspro app with 2 interfaces}{July 2019}{}

\begin{itemize}
%\item prend la photo du répondant
\item This app writes and interrogates a database in real time (while the survey takes place).
\item It generates a report with percentages of responses vs. non-responses.
\end{itemize}










\cvsection{Interests and hobbies}
Cinema , Painting, Arts, Camping




















\end{document}
